\documentclass[english,letterpaper]{book}
\usepackage{times}
\usepackage[T1]{fontenc}
\usepackage[latin1]{inputenc}
\usepackage{longtable}
\usepackage{amsmath}
\usepackage{amssymb}
\usepackage{floatflt}

\usepackage{fancyhdr}

\usepackage{bookman}

\pagestyle{fancy}

%\lhead{}
%\chead{}
\rhead{}

\lfoot{}
\cfoot{\thepage}
\rfoot{KOW Lib 1.0b}


\newcommand\Ref[1]{\textsection\ref{#1} (page~\pageref{#1})}

\usepackage{fancyvrb}
\usepackage{listings}

\usepackage{makeidx}
\makeindex

\IfFileExists{url.sty}{\usepackage{url}}
                      {\newcommand{\url}{\texttt}}

\makeatletter

\usepackage{babel}
\makeatother



%==========%
% HYPERREF %
%==========%
\usepackage[dvipdfm, bookmarks, colorlinks, breaklinks, pdftitle={KOW Lib General Purpose Libraries},
    pdfauthor={KOW Framework Project}]{hyperref}
\hypersetup{
	linkcolor=DarkSkyBlue,
	citecolor= DarkSkyBlue,
	filecolor= DarkSkyBlue,
	urlcolor= DarkSkyBlue
}



%========================%
% Listings Package Setup %
%========================%
\usepackage{xcolor}
\usepackage{listings}


% COLORS (Tango)
\definecolor{LightButter}{rgb}{0.98,0.91,0.31}
\definecolor{LightOrange}{rgb}{0.98,0.68,0.24}
\definecolor{LightChocolate}{rgb}{0.91,0.72,0.43}
\definecolor{LightChameleon}{rgb}{0.54,0.88,0.20}
\definecolor{LightSkyBlue}{rgb}{0.45,0.62,0.81}
\definecolor{LightPlum}{rgb}{0.68,0.50,0.66}
\definecolor{LightScarletRed}{rgb}{0.93,0.16,0.16}
\definecolor{Butter}{rgb}{0.93,0.86,0.25}
\definecolor{Orange}{rgb}{0.96,0.47,0.00}
\definecolor{Chocolate}{rgb}{0.75,0.49,0.07}
\definecolor{Chameleon}{rgb}{0.45,0.82,0.09}
\definecolor{SkyBlue}{rgb}{0.20,0.39,0.64}
\definecolor{Plum}{rgb}{0.46,0.31,0.48}
\definecolor{ScarletRed}{rgb}{0.80,0.00,0.00}
\definecolor{DarkButter}{rgb}{0.77,0.62,0.00}
\definecolor{DarkOrange}{rgb}{0.80,0.36,0.00}
\definecolor{DarkChocolate}{rgb}{0.56,0.35,0.01}
\definecolor{DarkChameleon}{rgb}{0.30,0.60,0.02}
\definecolor{DarkSkyBlue}{rgb}{0.12,0.29,0.53}
\definecolor{DarkPlum}{rgb}{0.36,0.21,0.40}
\definecolor{DarkScarletRed}{rgb}{0.64,0.00,0.00}
\definecolor{Aluminium1}{rgb}{0.93,0.93,0.92}
\definecolor{Aluminium2}{rgb}{0.82,0.84,0.81}
\definecolor{Aluminium3}{rgb}{0.73,0.74,0.71}
\definecolor{Aluminium4}{rgb}{0.53,0.54,0.52}
\definecolor{Aluminium5}{rgb}{0.33,0.34,0.32}
\definecolor{Aluminium6}{rgb}{0.18,0.20,0.21}




\lstset{
	keywordstyle=[1]{\color{DarkSkyBlue}},
	keywordstyle=[2]{\color{DarkScarletRed}},
	keywordstyle=[3]{\bfseries},
	keywordstyle=[4]{\color{DarkPlum}},
	keywordstyle=[5]{\color{SkyBlue}},
	commentstyle={\color{Aluminium4}\small},
	stringstyle={\color{Chocolate}},
	tabsize=4,
	breaklines=true,
	basicstyle={\ttfamily\small},
	xleftmargin=21pt,
	xrightmargin=11pt,
	frame=single,
	rulecolor=\color{black!30},
	captionpos=b,
	framesep=10pt,
	framexleftmargin=18pt,
	numbers=none,
	numberstyle={\tiny},
	stepnumber=1,
	numbersep=15pt
}

\lstdefinelanguage{bash}{%
	morekeywords={alfa,and,array,begin,boolean,byte,case,char,const,div,%
	do,downto,else,end,false,file,for,function,get,goto,if,in,%
	integer,label,maxint,mod,new,not,of,or,pack,packed,page,program,%
	procedure,put,read,readln,real,record,repeat,reset,rewrite,set,%
	text,then,to,true,type,unpack,until,var,while,with,write,writeln,make,\$},%
	sensitive=true,%
	morecomment=[s]{(*}{*)},%
	morecomment=[s]{\{}{\}},%
	morestring=[d]{'}%
}


\lstdefinelanguage{Ada}{%
	morekeywords={alfa,and,array,begin,boolean,byte,case,char,const,div,%
	do,downto,else,end,false,file,for,function,get,goto,if,in,%
	integer,label,maxint,mod,new,not,of,or,pack,packed,page,program,%
	procedure,put,read,readln,real,record,repeat,reset,rewrite,set,%
	text,then,to,true,type,unpack,until,var,while,with,write,writeln},%
	sensitive=false,%
	morecomment=[s]{(*}{*)},%
	morecomment=[s]{\{}{\}},%
	morestring=[d]{'}%
}
\lstdefinelanguage{SQL}{%
	morekeywords={select,insert,table,from,into,create,delete,alfa,and,array,begin,boolean,byte,case,char,const,div,%
	do,downto,else,end,false,file,for,function,get,goto,if,in,%
	integer,label,maxint,mod,new,not,of,or,pack,packed,page,program,%
	procedure,put,read,readln,real,record,repeat,reset,rewrite,set,%
	text,then,to,true,type,unpack,until,var,while,with,write,writeln},%
	sensitive=false,%
	morecomment=[s]{(*}{*)},%
	morecomment=[s]{\{}{\}},%
	morestring=[d]{'},%
	caption=SQL%
}


\lstnewenvironment{bash}{\lstset{language=bash}}{}



%==============%
% The document %
%==============%


\begin{document}

\title{KOW Lib General Purpose Libraries}
\author{%
Copyright (c) 2007-2009, KOW Framework Project
}
\date{\today}
\maketitle

\tableofcontents{}
\listoftables
%\listoffigures

\chapter{Introduction}
KOW Lib has been idealized to fullfill the simple and Operating System dependent needs
of the KOW Framework project.



\section{The KOW Framework}
The KOW Framework project has as it main objective to provide means for RAD of Web
Applications (most notably applications in for process and resource management)
using the Ada 2005 programming language.

Ada is one of the most strong and yet fast languages out there. It has been used
for decades in big projects. One could say that if you have some code where bugs
could lead to disasters you should Ada.

Yet, this programming language in my own work experience (and I've worked with PHP,
Java, C, C++ and Python) leads to less bugs and a clearer design.

There is a big debate and several articles about why you should (or should not) use
Ada in your project. If you are in doubt if the language is the best choice for you,
please google about. This is out of the scope of this manual.


\section{Rationale}
The functionality on this library could have been implemented in other KOW Framework
libraries. For instance, the KOW\_Lib.Libraries described at \Ref{api:libraries} could have been
implemented in KOW\_Plug module (not available in the 1.0b release but already functional).

The reason we've chose a central place for this kind of library is to easy the work
of maintaining the build system and extend OS support.

All other KOW Framework projects should be Operating System neutral.



\section{Implementation goals}
The main goal of KOW\_Lib is to provide a simple and consistent set of libraries that
work really well on Windows an Unix-like operating systems.

By working really well we ment not only the Ada code but also the building system,
specially \emph{GNAT Project} setting the right linker arguments for given OS.

For instance, in Darwin (Mac OS X) shared libraries and Ada has proven not to be a sane
combination. We actually don't know yet why is that, but we do know shared libraries
simply won't work with recent GNAT versions (\~GNAT GCC 4.3). As the KOW Framework
requires Ada 2005 packages we must have sure our code play well with a GNAT
greater than 4.2 and the host Operating System.

\section{Supported OS}
\label{supported_os}
The library is distributed in source form only. You should be able to build it in
most systems with GNAT 4.3 or greater.

The systems where we tested it so far are:

\begin{floatingtable}{
   \begin{tabular}{lcc}
   System       &  Kernel/Version &  Compiler      \\
   \hline
   Linux        &  2.6.X          &  GNAT GCC 4.3  \\
   Windows      &  2003/XP/Vista  &  GNAT GPL 2008 \\
   MacOS X      &  10.5           &  GNAT GCC 4.2  \\
   \end{tabular}}
   \caption{Database Product Support}\label{t:DBSupport}
\end{floatingtable}


Though it should run fine in all linux flavours with a recent GNAT compiler please
let us know if you have used it with succes in an system that's not listed here.


\chapter{Getting the library}
\label{getting_it}
You can download it from http://framework.kow.com.br

\section{Compiling}
\label{getting_it:compiling}
Once you have downloaded the lastest release, simply do a:

\begin{bash}
$ tar zxf kowlib-${VERSION}.tar.gz
$ cd kowlib-${VERSION}
$ make
$
\end{bash}

If it doesn't fail it means you have built KOW Lib. 

That's it. No \emph{./configure}, no \emph{cmake}, no \emph{make setup}.

If it has failed, you most problably have an unsupported system. Please verify
at \Ref{supported_os}.

\section{Installing}
\label{getting_it:installing}


\subsection{Short Version}

\begin{bash}
$ make install
$
\end{bash}


\subsection{Long Version}
KOWLib installs files in 3 places:
\begin{enumerate}
	\item some folder for specs and bodies
	\item some folder for the GPR file
	\item some folder for both compiled library and .ali files
\end{enumerate}

But that's not the only thing. Also you can tweak the way the gpr file is created.
It's usefull when building scripts for automated download and installation (such as
Gentoo's portage).

To control all these variants you have some variables you can set.

\subsubsection{PREFIX}
This will set the root of installation. The default is \emph{/ust/local} but you
could set it to any valid path in your system.

\subsubsection{LIB}

\chapter{Source Code Organization}
\label{source_organization}

\chapter{Package Documentation}
\label{api}

\section{Calendar Functions}
\label{api:calendar}


\section{File System Operations}
\label{api:file_system}

\section{Dynamic Libraries Handling}
\label{api:libraries}

\section{Locales}
\label{api:locales}


\section{Basic Logging}
\label{api:log}

\section{String variables}
\label{api:replacer}

\section{String Utilities}
\label{api:string_util}

\section{Wide String Utilities}
\label{api:wide_string_util}

\section{Containers}
\label{api:containers}

We provide instanciation of generic container libraries for \emph{Unbounded\_String} and \emph{Wide\_Unbounded\_String} types.

Those instanciations can be found in the given files:

\lstinputlisting[language=Ada,numbers=left,title=KOW\_Lib.UString\_Ordered\_Maps]{../src/kow_lib-ustring_ordered_maps.ads}

\lstinputlisting[language=Ada,numbers=left,title=KOW\_Lib.UString\_Vectors]{../src/kow_lib-ustring_vectors.ads}

\lstinputlisting[language=Ada,numbers=left,title=KOW\_Lib.UWide\_String\_Ordered\_Maps]{../src/kow_lib-uwide_string_ordered_maps.ads}

\lstinputlisting[language=Ada,numbers=left,title=KOW\_Lib.UWide\_String\_Vectors]{../src/kow_lib-uwide_string_vectors.ads}




These packages are given for convenience only
\end{document}
